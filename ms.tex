%\documentclass{emulateapj}
\documentclass[letterpaper,12pt,preprint]{aastex}

% packages
\usepackage{amssymb,amsmath,amsbsy}
\usepackage{booktabs}
\usepackage[caption=false]{subfig}

% commands
\newcommand{\given}{\,|\,}
\newcommand{\dd}{\mathrm{d}}
\newcommand{\transpose}[1]{{#1}^{\mathsf{T}}}
\newcommand{\inverse}[1]{{#1}^{-1}}
\newcommand{\msun}{\mathrm{M}_\odot}

\begin{document}

\title{Will it blend: Is the Milky Way bar causing chaotic fanning of the Ophiuchus stream?}
\author{Adrian M. Price-Whelan\altaffilmark{\colum,\adrn}, +others}

% Affiliations
\newcommand{\colum}{1}
\newcommand{\adrn}{2}

\altaffiltext{\colum}{Department of Astronomy,
                      Columbia University,
                      550 W 120th St.,
                      New York, NY 10027, USA}
\altaffiltext{\adrn}{To whom correspondence should be addressed: adrn@astro.columbia.edu}

\begin{abstract}

% Context
The Ophiuchus stream was recently discovered as an over-density of main sequence stars in Pan-STARRS (PS1) photometry and was later detected in BHB stars using line-of-sight velocities: XX of the YY BHB stars around the sky position of the stream appear kinematically distinct from the background halo population.
A small number of stars (XX) near the ends of the stream have similarly distinct velocities but with much larger velocity and spatial dispersions compared to the short segment of the stream first identified in the PS1 data.
In recent work, we have shown that streams formed on chaotic orbits may display similar kinematic signatures---short segments of recently-disrupted stellar debris flanked by high-dispersion, low-density ``stream-fans'' at the ends of the stream.
The stream's proximity to the triaxial, time-dependent potential of the Galactic bar and the peculiar kinematics and morphology of the stream suggest that the stream may have formed on a chaotic orbit.
% Aims
Here we study whether the Galactic bar can lead to chaotic orbits near (in phase-space) to the Ophiuchus stream. 
% Methods
We construct a potential model for the Milky Way bar based on recent measurements of the density of XXX stars.
We then assess the level of chaos for orbits like that of the Ophiuchus stream and generate mock streams along these orbits that qualitatively reproduce the observed kinematics of the stream.
% Results
We find that the apparent shortness and nearby low-density, high-velocity-dispersion ...
% Conclusions
These results motivate efforts to obtain deep photometry of the region surrounding the Ophiuchus stream to detect the predicted low-surface-brightness fans of stellar debris.
The existence of or lack of stream fans may provide an interesting constraint on the triaxiality and pattern speed of the bar.

\end{abstract}

% \keywords{}

\section{Introduction}\label{sec:introduction}

* Observational work on the stream (Bernard, Sesar)

(a short projected length of $\approx$XX deg)

The stream is $\approx$XX kpc from the Sun at a Galactocentric radius and height $(R,z) \approx (XX, YY)~{\rm kpc}$.

much larger than the measured $\lesssim 1~{\rm km}~{\rm s}^{-1}$ dispersion of the 

* Summarize chaos work

* Why is Ophiuchus interesting?

- Shortness of the stream

- Possible fanned debris

- Discrepancy between dynamical / stellar population age of the stream

The Ophiuchus stream is located in a unique position relative to other known stellar tidal streams: Many streams have been found in the halo of the Milky Way at Galactocentric distances of XX--YY kpc. For this stream, it is therefore expected that the baryonic components of the Milky Way---the combination of the bulge, disk, and Galactic bar---should dominate the kinematic history of the progenitor system. Unfortunately, the density distribution or gravitational potential of the inner Galaxy is not well known.

Recent modeling fit for the orbit, found discrepancy between dynamical age and age of stellar populations \citep{sesar15}, however they assume an axisymmetric disk plus spherical bulge and halo. If the Galactic bar contributes significantly to the density distribution of the inner Galaxy, the orbit of the cluster could be entirely different.

\section{Methods}\label{sec:method}

Our goal is to assess whether the Galactic bar can produce significant numbers of chaotic orbits in the inner Galaxy and see if the Ophiuchus stream could plausibly be on a strongly chaotic orbit. If chaos is important for this stream, we would then like to show that chaotic stream-fanning can reproduce the observed morphology and velocity signatures of ... In this section, we describe the methods we will use to detect and quantify the strength of chaos for orbits and then explain the method we will use to generate mock tidal streams.

\subsection{Potential model}\label{sec:potential}

We use ...
[Bulge from Portail et a. (2015), bar from Wegg and Gerhard (2015)?]

Though these density distributions are analytic and relatively simple, the potential and accelerations must be computed numerically. Rather than solve Poisson's equation at every evaluation of the potential or forces, we instead use the Self-Consistent Field (SCF) method \citep{hernquist92} to expand the density and potential on a series of orthogonal potential-density pairs. This method has been previously used to represent the potential from the Galactic bar \citep{zhao96, wangzhao12}. The advantage to this method is that the expansion coefficients (expensive integrals) only need to be computed once.

We follow the formalism introduced in \cite{hernquist92} and used in \cite{lowing11}: 


\subsection{Mock streams}\label{sec:mocks}

by releasing ensembles of test particles when the progenitor orbit crosses pericenter.

\section{Results}

\subsection{The degree of chaos}\label{sec:results1}

\subsection{Reproducing the observed properties of the stream}\label{sec:results2}

\section{Conclusions}\label{sec:conclusions}

\acknowledgements
APW is supported by a National Science Foundation Graduate Research Fellowship under Grant No.\ 11-44155.
APW acknowledges the ... MPIA ...
This research made use of Astropy, a community-developed core Python package for Astronomy \citep{astropy13}.
This work additionally relied on Columbia University's \emph{Yeti} compute cluster, and we acknowledge the Columbia HPC support staff for assistance. \\

%\bibliographystyle{apj}
%\bibliography{refs}

\end{document}
