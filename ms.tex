%\documentclass{emulateapj}
\documentclass[letterpaper,12pt,preprint]{aastex}

% packages
\usepackage{amssymb,amsmath,amsbsy}
\usepackage{booktabs}
\usepackage[caption=false]{subfig}

% commands
\newcommand{\given}{\,|\,}
\newcommand{\dd}{\mathrm{d}}
\newcommand{\transpose}[1]{{#1}^{\mathsf{T}}}
\newcommand{\inverse}[1]{{#1}^{-1}}
\newcommand{\msun}{\mathrm{M}_\odot}

\newcommand{\project}[1]{\textsl{#1}}

% TO DO
\newcommand{\todo}[2]{{\color{red} TODO: (\MakeUppercase{#1}) #2}}

\begin{document}

\title{}
\author{Adrian M. Price-Whelan\altaffilmark{\colum,\adrn},
            Branimir Sesar\altaffilmark{\mpia},
            Sarah Pearson\altaffilmark{\colum},
            Kathryn V. Johnston\altaffilmark{\colum},
            Hans-Walter Rix\altaffilmark{\mpia},
}

% Affiliations
\newcommand{\colum}{1}
\newcommand{\adrn}{2}
\newcommand{\mpia}{3}

\altaffiltext{\colum}{Department of Astronomy,
                      Columbia University,
                      550 W 120th St.,
                      New York, NY 10027, USA}
\altaffiltext{\adrn}{To whom correspondence should be addressed: adrn@astro.columbia.edu}
\altaffiltext{\mpia}{Max-Planck-Institut f\"ur Astronomie,
                     K\"onigstuhl 17, D-69117 Heidelberg, Germany}
                     
\begin{abstract}

% Context
The Ophiuchus stream was recently discovered as an over-density of main sequence stars in Pan-STARRS (PS1) photometry and was later detected in BHB stars using line-of-sight velocities: XX of the YY BHB stars around the sky position of the stream appear kinematically distinct from the background halo population.
A small number of stars (XX) near the ends of the stream have similarly distinct velocities but with much larger velocity and spatial dispersions compared to the short segment of the stream first identified in the PS1 data.
In recent work, we have shown that streams formed on chaotic orbits may display similar kinematic signatures---short segments of recently-disrupted stellar debris flanked by high-dispersion, low-density ``stream-fans'' at the ends of the stream.
The stream's proximity to the triaxial, time-dependent potential of the Galactic bar and the peculiar kinematics and morphology of the stream suggest that the stream may have formed on a chaotic orbit.
% Aims
Here we study whether the Galactic bar can lead to chaotic orbits near (in phase-space) to the Ophiuchus stream. 
% Methods
We construct a potential model for the Milky Way bar based on recent measurements of the density of XXX stars.
We then assess the level of chaos for orbits like that of the Ophiuchus stream and generate mock streams along these orbits that qualitatively reproduce the observed kinematics of the stream.
% Results
We find that the apparent shortness and nearby low-density, high-velocity-dispersion ...
% Conclusions
These results motivate efforts to obtain deep photometry of the region surrounding the Ophiuchus stream to detect the predicted low-surface-brightness fans of stellar debris.
The existence of or lack of stream fans may provide an interesting constraint on the triaxiality and pattern speed of the bar.

\end{abstract}

% \keywords{}

\section{Introduction}\label{sec:introduction}

The Ophiuchus stream \citep{bernard14, sesar15a} is a recently discovered stellar tidal stream that sits above the Galactic bulge at a Galactocentric radius and height $(R,z) \approx (1, 4.3)~{\rm kpc}$. All observational evidence suggests that the stream is a totally disrupted globular cluster: The stream stars have (1) a small positional dispersion orthogonal to the extended direction of the stream ($\approx$10 pc); (2) a small velocity dispersion $\approx$0.4 ${\rm km}~{\rm s}^{-1}$; (3) an old stellar population ($\approx$12 Gyr) estimated from isochrone fitting; and (4) no detectable over-density along the stream that could be  the progenitor system \citep[][hereafter S15a]{sesar15a}. 

The deprojected length of the visible part of the stream ($\approx$1.5 kpc) is short for the age of its stellar population. S15a fit an orbit to the kinematics of the stream stars in a static, axisymmetric model for the gravitational field of the inner Galaxy and ran N-body simulations of globular clusters on this orbit. S15a find that---on this orbit---the observed portion of the stream must have been formed in the last $\lesssim$400 Myr for the stream to remain as short as it is observed. This dynamical age is at odds with the old stellar population. More recently, a significant observational effort has followed-up nearly all BHB star candidates within XX deg of the stream to obtain radial velocities to identify more stream members \citep{sesar15b}. The stream has a very distinct and large line-of-sight velocity ($\approx$290 ${\rm km}~{\rm s}^{-1}$) and is therefore easily detected above the background halo population. Of the XX stars followed up, six have velocities significantly discrepant with the halo population. Of these six stars, two lie within the previously detected extent of the stream and have velocities consistent with the previously identified stream members. The remaining four stars lie close to an extrapolation of the stream, have velocities much larger than halo stars in this region ($v_{\rm los}>230~{\rm km}~{\rm s}^{-1}$), but have a velocity dispersion $\approx$75 times the measured internal velocity dispersion of the stream stars and positional spread much larger than the orthogonal positional dispersion of the main part of the stream. These low-density, high-dispersion features were not modeled in S15a and N-body simulations on the orbit fit from the previous work do not  reproduce these features.

It is not surprising that the orbit fit in the static, axisymmetric potential model used in S15a leads to a dynamical puzzle and failed to predict the new features: it is now well-known that the Galactic bulge contains a triaxial, bar-like structure several kpc in size \citep[e.g.,][]{blitzXX, wegg13, MANY}. Given the proximity of the stream to the center of the Galaxy, the time-dependent, triaxial potential of the Galactic bar must be taken into account when modeling the orbit of the progenitor system. The presence of a bar-like perturbation to the potential will change the possible orbit of the stream progenitor and will drastically change the resonance structure of orbits in the inner galaxy \citep{athanassoula, MANY}, potentially introducing a significant number of chaotic orbits in this region.

Recent work has shown that chaos can dramatically alter the density evolution of tidal streams. Along certain chaotic orbits, the stream stars will spread much faster in 3D position than from ordinary phase-mixing and, depending on the orbital phase at which the stream is observed, may develop large, low-density blobs of stars at the ends of a stream \citep{apw15-chaos}. [...]

In this work, we study whether stream-fanning---chaotic or simply from density evolution in a triaxial, time-dependent potential---can explain the observed properties of the Ophiuchus stream. In Section~\ref{sec:method} we describe the methods used in this work: in Section~\ref{sec:potential} we describe the models we use for the gravitational potential of the Galaxy, in Section~\ref{sec:mocks} we explain the simple method we use to generate mock streams, and in Section~\ref{sec:chaos-indicator} we briefly describe the methods we use for detecting and characterizing chaotic orbits. In Section~\ref{sec:results1} we discuss the differences in the orbit structure between a static, axisymmetric potential model and a model with a time-dependent bar potential, and in Section~\ref{sec:results2} we use mock streams to argue that [chaotic stream-fanning is a plausible explanation for the observational peculiarities of the Ophiuchus stream.] We conclude in Section~\ref{sec:conclusions}.

\section{Methods}\label{sec:method}

Our goal is to (1) assess whether the Galactic bar can produce significant numbers of chaotic orbits in the inner Galaxy; (2) see if the Ophiuchus stream could plausibly be on a strongly chaotic orbit; and (3) if chaotic density evolution can explain the apparent shortness of the stream and low-density, high-dispersion ends. In this section, we describe the potential model we use for the galaxy and describe the methods we use to detect and quantify the strength of chaos for orbits. We then describe a simplified version of the \emph{Streakline} method \citep{kuepper12} that we use to generate mock streams.

\subsection{Potential models}\label{sec:potential}

To integrate orbits and to compute chaos indicators we must choose a gravitational potential model to represent the potential of the Milky Way. The key feature of the potential that we would like to capture is the time-dependence and triaxiality of the Galactic bar. We construct two parametrized potential models: one with a triaxial, time-dependent bar and one with a purely spherical bulge, both added to simple models for the disk and halo of the Milky Way. Recent work has mapped the 3D density distribution of the region around the bulge using number counts of red clump stars and have measured the bar angle and axis ratios of the bar structure \citep{wegg13}. The authors do not fit an analytic density model to the data and do not provide enough information about the density field to reconstruct the full-space stellar density distribution of the bulge. Further, this is not a dynamical model for the bar and therefore may not account for all of the mass but rather provides information about the number density of stars in the Galactic bar. \cite{portail15}

We instead use an earlier model that uses a basis function expansion (BFE) to represent the potential and density of the bar \citep[][hereafter W12]{wang12}. The authors provide a table of expansion coefficients for a low-order expansion of the triaxial bar density used in \citet{dwek95}. The BFE formalism used in W12 neglects the sine expansion terms because of the symmetries of the triaxial density; the coefficients presented in W12 are for just the cosine terms (the $A_{lm}$ in \citet{hernquist92} or the $S_{nlm}$ in \citet{lowing11}). We have implemented the BFE computation of the potential, density, and gradient of the potential in C and Python and the code is publicly available on \project{GitHub}.\footnote{\url{https://github.com/adrn/biff}} The BFE representation effectively fixes the axis ratios of the bar (the exponential scale lengths along the three axes of the bar were adopted from \cite{dwek95} when the expansion coefficients were calculated in W12), but the mass of the bar, pattern speed, and angle with respect to the Galactic $x$ axis are free parameters. 

In both models, we use a spherical Navarro-Frenk-White potential for the halo \citep{nfw96}, a Miyamoto-Nagai potential for the disk \citep{miyamoto75}, and a Hernquist spheroid for the central spherical component of the bulge \citep{hernquist90}. Model A consists of just these three components and the total potential is therefore static and axisymmetric. Models B and C include the BFE bar potential described above: Model B contains a light bar and Model C contains a heavy bar. Except for the pattern speed and bar angle (which are varied), the parameter values for all potential models are given in Table~\ref{tbl:potential-params}. In all cases, the parameter values were chosen to qualitatively match the rotation curve of \citet{bovy12}

\begin{table*}[ht]
\begin{center}
	\begin{tabular}{ c  c | c  c}
	         \toprule
	         \multicolumn{2}{c|}{\bf Model A} & \multicolumn{2}{c}{\bf Model B} \\
	         Parameter & Value & Parameter & Value \\\toprule
		{\it Disk} &  &  & \\
		a & b & c & d\\
		a & b & c & d\\\midrule
		a & b & c & d\\
		\bottomrule
		\end{tabular}
	\caption{ \label{tbl:potential}}
\end{center}
\end{table*}


\subsection{Detecting and characterizing chaos with orbital frequency diffusion}\label{sec:chaos-indicator}

\subsection{Generating mock streams}\label{sec:mocks}

by releasing ensembles of test particles when the progenitor orbit crosses pericenter.


\section{Results}

\subsection{The degree of chaos}\label{sec:results1}

\subsection{Reproducing the observed properties of the stream}\label{sec:results2}

\section{Conclusions}\label{sec:conclusions}

\acknowledgements
APW is supported by a National Science Foundation Graduate Research Fellowship under Grant No.\ 11-44155.
APW acknowledges the staff at the MPIA for...
This research made use of Astropy, a community-developed core Python package for Astronomy \citep{astropy13}.
This work additionally relied on Columbia University's \emph{Yeti} compute cluster, and we acknowledge the Columbia HPC support staff for assistance. \\

\bibliographystyle{apj}
\bibliography{refs}

\end{document}
