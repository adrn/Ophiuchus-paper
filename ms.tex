%\documentclass{emulateapj}
\documentclass[letterpaper,12pt,preprint]{aastex}

% packages
\usepackage{amssymb,amsmath,amsbsy}
\usepackage{booktabs}
\usepackage[caption=false]{subfig}

% commands
\newcommand{\given}{\,|\,}
\newcommand{\dd}{\mathrm{d}}
\newcommand{\transpose}[1]{{#1}^{\mathsf{T}}}
\newcommand{\inverse}[1]{{#1}^{-1}}
\newcommand{\msun}{\mathrm{M}_\odot}

\begin{document}

\title{Will it blend: Is the Milky Way bar causing chaotic fanning of the Ophiuchus stream?}
\author{Adrian M. Price-Whelan\altaffilmark{\colum,\adrn}, +others}

% Affiliations
\newcommand{\colum}{1}
\newcommand{\adrn}{2}

\altaffiltext{\colum}{Department of Astronomy,
                      Columbia University,
                      550 W 120th St.,
                      New York, NY 10027, USA}
\altaffiltext{\adrn}{To whom correspondence should be addressed: adrn@astro.columbia.edu}

\begin{abstract}

% Context
The Ophiuchus stream was recently discovered as an over-density of main sequence stars in Pan-STARRS (PS1) photometry and was later detected in BHB stars using line-of-sight velocities: XX of the YY BHB stars around the sky position of the stream appear kinematically distinct from the background halo population.
A small number of stars (XX) near the ends of the stream have similarly distinct velocities but with much larger velocity and spatial dispersions compared to the short segment of the stream first identified in the PS1 data.
In recent work, we have shown that streams formed on chaotic orbits may display similar kinematic signatures---short segments of recently-disrupted stellar debris flanked by high-dispersion, low-density ``stream-fans'' at the ends of the stream.
The stream's proximity to the triaxial, time-dependent potential of the Galactic bar and the peculiar kinematics and morphology of the stream suggest that the stream may have formed on a chaotic orbit.
% Aims
Here we study whether the Galactic bar can lead to chaotic orbits near (in phase-space) to the Ophiuchus stream. 
% Methods
We construct a potential model for the Milky Way bar based on recent measurements of the density of XXX stars.
We then assess the level of chaos for orbits like that of the Ophiuchus stream and generate mock streams along these orbits that qualitatively reproduce the observed kinematics of the stream.
% Results
We find that the apparent shortness and nearby low-density, high-velocity-dispersion ...
% Conclusions
These results motivate efforts to obtain deep photometry of the region surrounding the Ophiuchus stream to detect the predicted low-surface-brightness fans of stellar debris.
The existence of or lack of stream fans may provide an interesting constraint on the triaxiality and pattern speed of the bar.

\end{abstract}

% \keywords{}

\section{Introduction}\label{sec:introduction}

(a short projected length of $\approx$XX deg)

The stream is $\approx$XX kpc from the Sun at a Galactocentric radius and height $(R,z) \approx (XX, YY)~{\rm kpc}$.

much larger than the measured $\lesssim 1~{\rm km}~{\rm s}^{-1}$ dispersion of the 

by releasing ensembles of test particles when the progenitor orbit crosses pericenter.

\section{Methods}\label{sec:method}

\section{Potential model}\label{sec:potential}

The three-dimensional shape and density distribution of the inner galaxy is not well known. 

\section{Conclusions}\label{sec:conclusions}

\acknowledgements
APW is supported by a National Science Foundation Graduate Research Fellowship under Grant No.\ 11-44155.
This work was supported in part by the National Science Foundation under Grant No. PHYS-1066293.
This research made use of Astropy, a community-developed core Python package for Astronomy \citep{astropy13}.
This work additionally relied on Columbia University's \emph{Hotfoot} and \emph{Yeti} compute clusters, and we acknowledge the Columbia HPC support staff for assistance. \\

%\bibliographystyle{apj}
%\bibliography{refs}

\end{document}
